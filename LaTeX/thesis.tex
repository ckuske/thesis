\documentclass[12pt]{amsart}


%%%%%%%%%%%%%%%% Make appropriate substitutions here %%%%%%%%%%%%
\newcommand{\chairfaculty}{Dr. Michael Soltys}
\newcommand{\firstfaculty}{Committee Member's name}
%Uncomment the next line as needed.
%\newcommand{\secondfaculty}{Second Committee Member's name}
\newcommand{\univefaculty}{Dr. Gary A. Berg}
\def\thesistitle{Derivation of Consistent Pairwise Matrices}
\def\name{Chris Kuske}
\newif \ifshort

\shorttrue %Leave this command here if your committee is short (2 people), comment it out if your committee is long (3 people).

\newcommand{\committee}[1]{\ifshort {\committeeshort} \else {\committeelong} \fi}
\newcommand{\committeeshort}{\vspace*{2.13in} \begin{tabular}{ll}
   \multicolumn{2}{c}{\hspace*{2.9cm} APPROVED FOR THE COMPUTER SCIENCE PROGRAM}\\[10mm]
   \multicolumn{2}{c}{\hspace*{1.65cm}\rule{4.5in}{.01in}}\\[-4mm]
   \hspace*{3cm}\chairfaculty, Thesis Advisor \hspace*{0cm}&  Date\\[4mm]
   \multicolumn{2}{c}{\hspace*{1.65cm}\rule{4.5in}{.01in}}\\[-4mm]
   \hspace*{3cm}\firstfaculty, Thesis Committee \hspace*{0cm}&  Date\\[17mm]
   \multicolumn{2}{c}{\hspace*{6.15cm} APPROVED FOR THE UNIVERSITY}\\[6mm]
   \multicolumn{2}{c}{\hspace*{1.9cm}\rule{4.5in}{.01in}}\\[-4mm]
   \hspace*{3cm}\univefaculty, AVP Extended University \hspace*{0cm}&  Date\\
 \end{tabular}}
\newcommand{\committeelong}{\vspace*{1.25in} \begin{tabular}{ll}
   \multicolumn{2}{c}{\hspace*{2.9cm} APPROVED FOR THE COMPUTER SCIENCE PROGRAM}\\[10mm]
   \multicolumn{2}{c}{\hspace*{1.65cm}\rule{4.5in}{.01in}}\\[-4mm]
   \hspace*{3cm}\chairfaculty, Thesis Advisor \hspace*{0cm}&  Date\\[4mm]
   \multicolumn{2}{c}{\hspace*{1.65cm}\rule{4.5in}{.01in}}\\[-4mm]
   \hspace*{3cm}\firstfaculty, Thesis Committee \hspace*{0cm}&  Date\\[17mm]
   \multicolumn{2}{c}{\hspace*{1.65cm}\rule{4.5in}{.01in}}\\[-4mm]
   \hspace*{3cm}\seconfaculty, Thesis Committee  &  Date\\[20mm]
   \multicolumn{2}{c}{\hspace*{6.15cm} APPROVED FOR THE UNIVERSITY}\\[6mm]
   \multicolumn{2}{c}{\hspace*{1.9cm}\rule{4.5in}{.01in}}\\[-4mm]
   \hspace*{3cm}\univefaculty, AVP Extended University \hspace*{0cm}&  Date\\
 \end{tabular}}



\usepackage{amssymb,mathtools,latexsym,amsmath,amsthm,amsfonts,graphics}
\newcommand\Month{%
\ifcase\number\month\relax%
\or January\or February\or March\or April\or May\or June%
\or July\or August\or September\or October\or November\or December\fi}
\newcommand{\R}{\mathbb{R}}
\newcommand{\C}{\mathbb{C}}
\newcommand{\N}{\mathbb{N}}
\newtheorem{problem}{Problem}
\newtheorem{theorem}{Theorem}
\newtheorem{lemma}{Lemma}
\newtheorem{example}{Example}
\theoremstyle{definition}

%thesis margins
\setlength{\paperheight}{11.5in} \setlength{\headsep}{0in}
\setlength{\topmargin}{-0.4375in} \setlength{\headheight}{0in}
\setlength{\voffset}{1in} \setlength{\oddsidemargin}{0.7in}
\setlength{\evensidemargin}{0.7in} \setlength{\textheight}{7.9in}
\setlength{\textwidth}{5.3in} \setlength{\footskip}{0.75in}
\linespread{2.0}

\makeatletter
\def\subsubsection{\@startsection{subsubsection}{3}%
  \z@{.5\linespacing\@plus.7\linespacing}{.1\linespacing}%
  {\normalfont\itshape}}
\makeatother

\begin{document}
 \newpage
This page is here to make the page numbers come out correctly.

Do not print this page.
\newpage

\pagestyle{empty} \pagenumbering{roman}

\vspace*{.5in}

\begin{center}
\LARGE\it \thesistitle
\end{center}

\vskip 1in

\begin{center}
A Thesis Presented to \\
The Faculty of the Computer Science Program \\
California State University Channel Islands
\end{center}

\vskip .8in

\begin{center}
In (Partial) Fulfillment \\
of the Requirements for the Degree \\
Masters of Science
\end{center}

\vskip .8in

\begin{center}
by
\end{center}
\begin{center}
\name
\end{center}

\begin{center}
 \Month, \number\year
\end{center}
\newpage

%%%%%%%%%%%%%%%%%%%%%%%%%%%%%%%%%%%%%%%%%%%%%%%%%%%%
% copyright notice
% this is optional

\newpage

\setlength{\topmargin}{2cm}
%\setlength{\textheight}{25.5cm}

\vspace*{6in}

\begin{center}

\copyright{} \number\year\\
\name\\
ALL RIGHTS RESERVED


\end{center}
%%%%%%%%%%%%%%%%%%%%%%%%%%%%%%%%%%%%%%%%%%%%%%%%%%%%%%%%%%%%%%%%%%%%%%%%%
% approval page

\newpage
 \setlength{\paperheight}{13in}
 \setlength{\topmargin}{1.25cm}

%Insert your name in at the end of the next line
{\center\emph{Signature page for the Masters in Computer Science Thesis of Christopher Kuske }}

\committee


%%%%%%%%%%%%%%%%%%%%%%%%%%%%%%%%%%%%%%%%%%%%%%%%%%%%%%%%%%%%%%%%%
% A dedication is optional. Edit this section to make it your own or delete it.
 \newpage
 \setlength{\topmargin}{-0.4375in}
 \setlength{\paperheight}{11.5in}

\begin{center}
\vspace*{60mm}

To my family, in gratitude for their encouragement and support.

\end{center}

%%%%%%%%%%%%%%%%%%%%%%%%%%%%%%%%%%%%%%%%%%%%%%%%%%%%%%%%%%%%%%%%%%%%%%%%%%%
% Acknowledgements are also optional. Edit this section to make it your own or delete it.

\newpage

\begin{center}
\vspace*{50mm}
{\bf Acknowledgements}
\end{center}

This dissertation could not have been written without the constant advice and encouragement of my advisor, Professor Michael Soltys.
Many thanks must also go to the members of my committee, name and name. 
%%%%%%%%%%%%%%%%%%%%%%%%%%%%%%%%%%%%%%%%%%%%%%%%%%%%%%%%%%%%%%%%%%%%%%%%%
% abstract

\newpage
\pagestyle{plain}
 

{\bf Abstract}

\bigskip

\thesistitle

by
\name \bigskip
\begin{quote}

This thesis will give an overview of pairwise matrices and their properties. After this introduction, a summary of existing literature on Pairwise Matrices will follow.  

A method of generating a consistent Pairwise Matrix from an inconsistent matrix will be presented, along with a method to find a consistent matrix that is as close to the original inconsistent matrix as possible using a calculated distance.

After this methodology has been described, an analysis of the results and further work to be done will follow.
\end{quote} 

%%% Contents Page: LaTeX will figure this out for you as long as you use Section and Subsection commands.%%%
\newpage
\tableofcontents
\newpage
% List of Figures
\newpage

%\newpage

\pagenumbering{arabic}

%%%%%%%%%%%%%%% OK, this is where your content goes! %%%%%%%%%%%%%%%%%%%
\section{Introduction}

Over the past few decades, pairwise comparisons and the Analytical Hierarchical Process (\textit{AHP}) have given decision makers a new set of tools that empower them to make more informed decisions.  AHP uses matrices to help rank the evaluation criteria based on importance.  This type of problem is also knows as Multiple Attribute Decision Making \textit{(MADM)}.

In pairwise matrices, each criterion has a relative rank.  Consider the following matrix:
\begin{center}
$
\begin{bmatrix}
    1 & x_{12} & x_{13} & x_{1n} \\
    x_{21} & 1 & x_{23} & x_{2n} \\
    x_{31} & x_{32} & 1 & x_{33} \\
    x_{d1} & x_{d2} & x_{d3} & 1
\end{bmatrix}
$
\end{center}

When decision makers are trying to make their evaluation(s), they will often bring in subject matter experts \textit{(SMEs)} to help develop the relative rankings of how one preference should be ranked compared to another.  When experts define their preferences in a pairwise matrix, they often generate matrices that do not meet the criteria for consistency.

\newpage
A pairwise comparison matrix $A$ has the following properties:
\begin{itemize}
\item $A$ is square ($n$x$n$).
\item All elements on the diagonal of $A$ have a value of 1.
\item $A$ has the property where each element $a_{ij}$ has an element that is the reciprocal, located at $a_{ji}$.
\item $A$ is consistent.  This means that for each element $a_{ij}$, $a_{ij}=a_{ik}*a_{kj}$.  For all $i,j,k$ $\{1,\dots,n\}$. If $a_{ij}!=a_{ik}*a_{kj}$, the matrix is not consistent.
\end{itemize}

\cite{saaty2000fundamentals}

$
\begin{bmatrix}
    x_{11} & x_{12} & x_{13} & \dots  & x_{1n} \\
    x_{21} & x_{22} & x_{23} & \dots  & x_{2n} \\
    \vdots & \vdots & \vdots & \ddots & \vdots \\
    x_{d1} & x_{d2} & x_{d3} & \dots  & x_{dn}
\end{bmatrix}
$

\subsection{Notation and Terminology}
\begin{itemize}
\item \textbf{Consistency:} A rectangular array of numbers.  The numbers in the matrix are called \textit{entries} or \textit{elements}.  The horizontal lines of entries are \textit{rows} while the vertical lines of entries are \textit{columns}.

\item \textbf{Matrix:} A rectangular array of numbers.  The numbers in the matrix are called \textit{entries} or \textit{elements}.  The horizontal lines of entries are \textit{rows} while the vertical lines of entries are \textit{columns}.

\item \textbf{Pairwise Matrix:} A Pairwise Matrix is defined as a square matrix

\item \textbf{Reciprocal:} The number 1/$x$, which multiplied by $x$ gives the product of 1.
\end{itemize}

\section{Literature Review}

%
%\subsection{Subsection Title}
Saaty

\section{Methodology}

\subsection{Formation of Consistent Matrix}
Dude
\subsection{Example}

\subsection{Distance Calculation}
%
%\begin{theorem}\label{T:H}
%    The number of maximum inconsistencies ($Mi$) in a \textit{PRM} (positive reciprocal matrix) is proportional to the size of matrix $M$.  The number of inconsistencies is determined by the following formula:
%    
%    \begin{equation}Mi = \frac{n(n+1)}{2}\end{equation}
%\end{theorem}
%
%\begin{proof}
%    There are those who might leave this proof to the reader... to do so would be inappropriate.
%\end{proof}

If $M$ is consistent, any row or column of $M$ may be selected such that:

[$w_1$,$w_2$,...,$w_n$] = [$a_{11}$, $a_{21}$, $a_{31}$, ..., $a_n1$]

By consistency of $M$, it is also true that 
$a_{1n}$ = $a_{1i}$ * $a_{in}$

$a'_{in}$ := $\frac{w_i}{w_n}$

which when further decomposed, the following holds true:

$\frac{w_i}{w_n}$ = $\frac{a_{1i}}{a_{ni}}$ which is in turn is equivalent to $\frac{\frac{a_{1n}}{a_{in}}}{a_{ni}}$

$\frac{a_{1n}}{a_{in}*a{ni}} = 1$

which reduces to simply $a_{1n}$, since $\frac{a_{1n}}{1} = 1$

${a_{in}*a_{ni}}$ is always 1 via the properties of pairwise matrices.

Let $M'$ to be the $M'<W>$ where $\|M-W\|$ is smallest.



%In more than one dimension, the analogue to Hamburger's theorem has a slightly more complicated expression.  The full proof
%may be found in \cite{F}.



\section{Results}

Consider addressing what the next person to work in this area might tackle.

\section{Conclusion}

Here is more stuff where I prove that this paper has something worthwhile inside of it.

\bibliographystyle{acm}
\bibliography{thesis}



\end{document}
